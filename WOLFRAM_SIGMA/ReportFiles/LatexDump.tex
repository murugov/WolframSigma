\documentclass[a4paper,10pt]{extarticle}
\usepackage{icomma}
\usepackage{amsmath}
\usepackage[english,russian]{babel}
\usepackage{graphicx}
\graphicspath{{images/}}
\usepackage[export]{adjustbox}
\usepackage{subcaption}
\usepackage[T2A]{fontenc}
\usepackage[utf8]{inputenc}
\usepackage{amsfonts,amssymb,amsthm,mathtools}
\usepackage[paperheight=16cm,paperwidth=12cm]{geometry}
\usepackage{fancyhdr}
\usepackage{wrapfig}
\usepackage{caption}
\usepackage{titlesec}

\titleformat{\section}
  {\normalfont\fontsize{14}{16}\bfseries}{\thesection}{1em}{}

\begin{document}
\thispagestyle{empty}
\vspace*{80px}
\begingroup
\fontsize{16pt}{14pt}
\selectfont
\begin{center}
Московский физико-технический институт\\
\textbf{"Нахождение производной функции"}
\end{center}
\endgroup
\begin{center}
\normalsize
\textit{Муругов А.Р., студент 1 курса ФРКТ}\\
\today
\end{center}
\newpage
\pagestyle{fancy}
\fancyhead{}
\fancyhead[LO]{\text{Нахождение производной функции}\hfill}

\begingroup
\fontsize{16pt}{14pt}
\section{Цели работы:}
\normalsize
\begin{enumerate}
    \item Найти производную функции
\end{enumerate}
\endgroup

\begingroup
\fontsize{16pt}{14pt}
\renewcommand{\baselinestretch}{1.5}
\section{Оборудование:}
\normalsize
\begin{enumerate}
    \item Мой любимый Mac
\end{enumerate}
\endgroup
\begingroup
\renewcommand{\baselinestretch}{1.5}
\section{Теоретическая часть}
\normalsize
\[
{\left(-0.627917\right)} + {\left(-9.94059\right)} \cdot {x} + {13.9719} \cdot {x} ^ {2} + {871.053} \cdot {x} ^ {3} + {1256.14} \cdot {x} ^ {4} + {\left(-33484.4\right)} \cdot {x} ^ {5} + {\left(-72858.6\right)} \cdot {x} ^ {6} + {785745} \cdot {x} ^ {7}
\]

\endgroup

\begingroup
\fontsize{16pt}{14pt}
\renewcommand{\baselinestretch}{1.5}
\section{Список используемой литературы}
\normalsize
\begin{enumerate}
    \item А.Ю. Петрович "Лекции по математическому анализу"
    \item Л.Д. Кудрявцев "Сборник задач по математическому анализу"
    \item Мои записи с семинаров по матану
\end{enumerate}
\endgroup

\end{document}
